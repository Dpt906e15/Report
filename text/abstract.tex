Computational thinking is becoming a skill needed more and more in everyday life. This  to several countries introducing programming to children in primary school. However, it is questionable whether the tools and methods used are adequate for the task. We investigate the area of educational tools and environments in order to understand their capabilities in educating novices. We describe several languages and environments, and analyse a set of these, being Scratch, BlueJ and DrRacket. The analysis is based on a set of subjectively defined criteria, and is followed by a comparison of the tools. We conclude Scratch to be the tool best fitting for novices, as it is easily learned and gives a great understanding of the basics in computational thinking.