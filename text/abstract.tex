Computational thinking is becoming a skill needed more and more in everyday life. This leads to several countries introducing children to programming in primary school. However, it is questionable whether the tools and methods used are adequate for the task. We investigate the area of educational tools and environments to understand their capabilities in educating novices. We have described several languages and environments, and analysed a set of these, being Scratch, BlueJ and DrRacket. The analysis was based on a set of subjectively defined criteria, and was followed by a comparison of the tools. We concluded Scratch to be the tool best fitting for novices, as it is easily learned and gives a great understanding of the basics in computational thinking.