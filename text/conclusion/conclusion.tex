\chapter{Conclusion}
\label{chap:conclusion}

Through this report, we have analysed the field of novice programming and the problems regarding learning programming. We have investigated several languages for novices, and discussed their merits and shortcomings. There is a huge difference between visual and text based programming, which is addressed, where it was found that visual programming has an advantage in its ease of access and getting acquainted with, where the user for example do not use the keyboard for much. Text based programming, on the other hand, requires a better understanding of how a computer is operated, which could lead to a problem in the younger group of novices. We have analysed the field of teaching, and concluded that the teaching approach varies from country to country. The U.K. seems to have a good grasp on novice programming, where Denmark has some trouble, as the teaching is done by possibly unqualified teachers. The US have not made programming mandatory yet, but as an optional course, counting towards final grades. Furthermore, research is being made in how to incorporate programming in other courses.

We chose to continue the analysis with three novice languages and environments, being Scratch, BlueJ and DrRacket, for a language comparison. This was carried out as a subjective evaluation by the authors, with the intention of understanding the novice approach to programming. We analysed each language by a given set of criteria, which were criteria for evaluating programming languages in general, and for evaluating the interface design. Some sample pieces of code were made for comparison, which were made to match each of the paradigms representing the languages. After discussing each criteria for each language, with examples through the sample code, we compared the languages in each criteria. We concluded that a great difference lies in the targeting age. Scratch can work as a great introduction to programming down to the age of a primal school pupil. BlueJ serves as a great introduction to general purpose programming, but requires more effort and concentration, as well as external recourses for understanding Java. DrRacket can be hard, due to the different syntax. Furthermore, as with BlueJ, it needs external resources to function as a novice environment.

\chapter{Discussion}
\label{chap:discussion}

During the initial stage of our project, we spend a great deal of time and effort on narrowing down the specific problem we wanted to address in the research area of our choice. The field of programming technologies is quite big and varied, and it encompasses many research areas and avenues one can explore. In order to address  the raising demand for programmers worldwide, educational institutions spend more and more time and resources on properly educating people for such a monumental task. However, there is not a well established way of achieving this task, especially for people in the early stages of their lives. Through our research and analysis of the existing research domain we chose to focus on identifying what the difficulties are that young people face when they start learning programming and what the most common and severe mistakes they usually make are. The reason for this focus was the fact that the evidence for the design decisions of some programming languages was quite limited. The best case we found was an article by Stefik and Siebert \cite{QuorumRandomoPerl}, and even then it did not tell us a lot about the design decisions for making it novice friendly. Instead they compare its usefulness to other languages for novices. We tried to cover as much as possible from the research area by analysing both text-based and visual-based programming languages. 

There were several difficulties which we faced while working with our research question. First, we had difficulties identifying which might be the most severe problems pupils face when given programming tasks to solve. The research on that specific area is highly opinionated and there is not a single right and perfectly accurate answer how to achieve that. Therefore, we focused on researching the resources we considered most important for novices. Second, we spend considerable time on deciding which are the most relevant programming paradigms in the context of teaching programming and which educational languages are best fit for representing those paradigms, reflected by their usage by novices. Last but not least, we had to familiarize ourselves with each one of the three programming environments we have chosen in order to acquire a better overall understanding of their inner workings and to be able to give our subjective opinion as part of the analysis. Given that we all a background in imperative programming, we found DrRacket the hardest to understand due to the nature of functional programming.

We consider Scratch as a great choice for teaching programming to novices given its pedagogical value and learnability. It provides a fast way to implement different ideas without the need to have knowledge about the entirety of the platform. However, we are still conscious about its shortcomings and how novices can quickly outgrow the platform. Although the verbosity of the language is one of its main strengths, it is also one of its weaknesses since the code built mostly by blocks tends to get very big for medium-sized projects and impractical for larger ones, which might significantly reduce its readability. We are aware that this is an intentional design decision made by the Scratch team, since the purpose of the language is to teach the basic concepts of programming rather than be a fully-fledged programming tool for building programs. 

During our research we investigated several new programming languages which were mostly educational by design and tried to address the ease of use while programming. What most of these did was to make claims which they did not necessarily had the empirical data to support scientifically which also made it harder to reproduce their state results. 

Digital literacy is an essential component of a modern education, which can directly affect a particular country's economy. Currently, there is a growing shortage of people skilled in ICT and there is an estimate is that there will be 900,000 vacant jobs in Europe's ICT Secton by 2020 \cite{pretz14}. This is one of the main reasons why the introduction of programming as a mandatory course became a reality in Denmark in 2014, following the trend established in several European countries, starting with England. Based on their age, children are getting lessons on algorithms, coding in languages such as Scratch, and debugging programs. In Denmark however, at this point there are no dedicated teachers for teaching programming, but rather that role is filled in by the physics teachers in schools. This might mean that the learning experience which is provided might not be on a satisfactory level, but there is still no sufficient data to support that claim.

\subsection{Coding Pirates}
\label{sec:coding_pirates}
Even though the danish government might not have the best handle on how to incorporate programming into the school system (See \secref{chap:teaching}), there are another organization which tries to put programming in the hands of children. This organization is called Coding Pirates\footnote{https://codingpirates.dk/}, and have multiple departments in different cities in Denmark. We have visited one of these departments located in Aalborg, to get a hands on experience with how they approach teaching programming to children. This also resulted in a conversation with one of the board members of the organization, Magnus Toftdal Lund, where we talked about the intentions of the organization, his observation about the children learning to program, and how the other departments were doing, as the one we visited was just starting up.

At the event we split up and followed two groups, one for 7-10 year olds who worked with scratch and one for the 10+ year olds who worked with Unity. Sadly, the group working with Unity did not get further than installing the software. The Scratch group got a bit further, first learning to log in on the Scratch website and then having a basic introduction to Scratch. The introduction was based on making a sprite of a cat dance, by moving it forwards and backwards infinitely, and changing the background so it was more fitting for a dance party. We observed for a bit longer where we saw that already after the basic introduction, some of the children had started experimenting on their own, and was asking questions about how to do certain things. This was also the intent of the instructor as he made it very clear that the event was not school, but about doing what they wanted. The instructor and volunteers where just there to help them get started and help them if they encountered problems, beyond the basic introduction. They would of course also help with ideas for projects if the children had none of their own.

After observing the group we went to have a talk with Magnus. We first discussed how fast the children were learning and how long it took for them to get comfortable with Scratch. He stated that already on the first day, they try to experiment, as we also observed, and that they relatively quick got a good understanding of Scratch and how to manipulate different aspects of it. He also stated that they tried to help the children to get comfortable with Scratch quickly as they felt that it was quite limiting in what could be done without having to do a lot of workarounds, but a good introduction device none the less. The aim is to give them enough knowledge about programming to move on to ``Pygame''\footnote{http://pygame.org/hifi.html}, and if the children wanted to continue on from Pygame to 3D games, give them further knowledge to start using Unity. We tend to agree with his opinion about Scratch after we had our own experience with the environment, as stated earlier in this chapter.

The second part of the discussion was about the litterature on what obstacles children has when learning to program. We asked him about obstacles regarding syntax when the children moved on from Scratch to a text based programming language. He stated that some scientists would focus on obstacles which, in the big picture of things, produced relatively limited problems for novices, here among the syntax of text-based programming languages. In his experience syntax was something easily overcome by a little bit of experience, but did not specify further.

The third part of the discussion was about the philosophy they imployed when trying to teach children programming. He talked about that in university, when the students start learning to program, they are often introduced to the features of the language, and first then a problem where those features were applicable. He thought that it was not an efficient way of doing it, as the students might ask why e.g. integers, floats, double and so on, was needed. The philosophy that they imploy is the opposite, which focuses on presenting a problem where these features are needed to solve the problem. That way the children learn about the features knowing why, instead of ``just needing it to pass an exam''.

In the end he stated that Coding Pirates was not actually for learning to be a great programmer, but making the attendants experienced in reading and understanding code and making them able to tweak code, but not creating code from scratch. The main focus of the events becomes game design rather than programming when the attendants have a good enough knowledge base about programming. One of the observations that he mentioned was that the attendants will start to split up into smaller teams and specialize in certain aspects of game creation e.g. one who draws the sprites, one who does the programming, one who comes up with the rules, etc. This makes them able to handle most aspects of game creation for a project internally in the team.



\section{Further Works}
\label{sec:further_works}
Following the conclusion of Shane Markstrum\todo{ref needed} the programming language community, including us, need to provide some scientific data behind any claim made on the presentation of the language.
The common suggestion is to use the techniques from social sciences to gather data, but these techniques often require a number of participants in the thousands, to make the data significant enough.
This is usually beyond the resource available to the language developers, and far beyond our resources.
At the same time there has been a large push to make programming more accessible to the general public, which is especially evident with Scratch, which is built to be intuitive enough to be used without external assistance.
This leads us to the idea of using the usability test from human-computer interaction on programming language environments.
These tests are used to see where a program has problems with the users easily interacting with the system.
Since programming language environment can be seen as a program to allow the user to interact more easily with the language, this means the tests are applicable on them.
Furthermore usability tests only require a few participants to say a lot about the program making it a lot more accessible to developers, who in fact are already frequently using them to improve their programs.
These tests could both be used to analyse the existing languages as well as any new language that might develop.

Given that Scratch is such a great environment for novices, it makes sense to teach it as a first language.
However even with its extensions it is still limited in its application, and does not provide the ability to start making programs for all sorts of applications like a general purpose language can do.
Furthermore Scratch's appearance is too different from a general purpose language to facilitate an easy transition to those.
This means a potential project could be to make a programming environment that makes it easier to transition from Scratch to a general purpose language like C\#.
Of course this would require figuring out which parts of the transition are difficult, and trying to break those into smaller more easily learned steps.

Alternatively one could try to make a language with general purpose applications, but which tries to replicate the intuitiveness from Scratch's interface to make it more accessible.
This could be a great challenge since a lot of Scratch's intuitiveness comes from its visual interface, which might not be an easy fit for a programming language designed for larger and more varied programs.

%I kept out the idea about a tutorial tool for scratch to help people less comfortable with the quick sandbox since it kinda already is in Scratch.