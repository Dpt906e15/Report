\section{Further Works}
\label{sec:further_works}
Following the conclusion of Staking Claims\todo{ref needed} the programming language community, including us, need to provide some scientific data behind any claim made on the presentation of the language.
The common suggestion is to use the techniques from social sciences to gather data, but these techniques often require a number of participants in the thousands, to make the data quantitative enough.
This is usually beyond the resource available to the language developers, and far beyond our resources.
At the same time there has been a large push to make programming more accessible to the general public, which is especially evident with Scratch, which is built to be intuitive enough to be used without external assistance.
This leads us to the idea of using the usability test from human-computer interaction on programming language environments.
These tests are used to see where a program has problems with the affordance that allows users to easily interact with the system.
Since programming language environment can be seen as a program to allow the user to interact more easily with the language, this means the tests are applicable on them.
Furthermore usability tests only require a few participants to say a lot about the program making it a lot more accessible to developers, who in fact are already frequently using them to improve their programs.
These tests could both be used to analyse the existing languages as well as any new language we might develop.

Given that Scratch is such a great environment for novices, it makes sense to teach it as a first language.
However even with its extensions it is still limited in its application, and does not provide the ability to start making programs for any sort of applications like a general purpose language can do.
Furthermore Scratch's appearance is too different from a general purpose language to facilitate an easy transition to those.
This means a potential project could be to make a programming environment that makes it easier to transition from Scratch to a general purpose language like C\#.
Of course this would require figuring out which parts of the transition are difficult, and trying to break those into smaller more easily learned steps.

Alternatively one could try to make a language with general purpose applications, but which tries to replicate the intuitiveness from Scratch's interface to make it more accessible.
This could be a great challenge since a lot of Scratch's intuitiveness comes from its visual interface, which might not be an easy fit for a programming language designed for larger and more varied programs.

%I kept out the idea about a tutorial tool for scratch to help people less comfortable with the quick sandbox since it kinda already is in Scratch.