\subsection{Coding Pirates}
\label{sec:coding_pirates}
Even though the Danish government might not have the best handle on how to incorporate programming into the school system (See \secref{chap:teaching}), there is another organization which tries to put programming in the hands of children. This organization is called Coding Pirates\footnote{https://codingpirates.dk/}, and have multiple departments in different cities in Denmark. We have visited one of these departments located in Aalborg, to get a hands on experience with how they approach teaching programming to children. This also resulted in a conversation with one of the board members of the organization, Magnus Toftdal Lund. We talked about the intentions of the organization, his observations about the children learning to program, and how the other departments were doing, as the one we visited was just starting up.

At the event we split up and followed two groups, one for 7-10 year olds who worked with scratch and one for the 10+ year olds who worked with Unity. Sadly, the group working with Unity did not get further than installing the software. The Scratch group got a bit further, first learning to log in on the Scratch website and then having a basic introduction to Scratch. The introduction was based on making a sprite of a cat dance, by moving it forwards and backwards infinitely, and changing the background so it was more fitting for a dance party. We observed for a bit longer where we saw that already after the basic introduction, some of the children had started experimenting on their own, and were asking questions about how to do certain things. This was also the intent of the instructor present, as he made it very clear that the event was not school, but about doing what they wanted. The instructor and volunteers where just there to help them get started and help them if they encountered problems, beyond the basic introduction. They would of course also help with ideas for projects if the children had none of their own.

After observing the group, we went to have a talk with Magnus. We first discussed how fast the children were learning and how long it took for them to get comfortable with Scratch. He stated that already on the first day, they try to experiment, as we also observed, and that they relatively quickly obtained a good understanding of Scratch and how to manipulate different aspects of it. He also stated that they tried to help the children to get comfortable with Scratch quickly as they felt that it was quite limiting in what could be done without having to do a lot of workarounds, but a good introduction device none the less. The aim is to give them enough knowledge about programming to move on to ``Pygame''\footnote{http://pygame.org/hifi.html}, and if the children wanted to continue on from Pygame to 3D games, give them further knowledge to start using Unity. We tend to agree with his opinion about Scratch after we had our own experience with the environment, as stated earlier in this chapter.

The second part of the discussion was about the litterature on what obstacles children have when learning to program. We asked him about obstacles regarding syntax when the children moved on from Scratch to a text-based programming language. He stated that some scientists would focus on obstacles which, in the big picture, produced relatively limited problems for novices, here among the syntax of text-based programming languages. In his experience, syntax was something easily overcome by a little bit of experience, but did not specify further.

The third part of the discussion was about the philosophy they imployed when trying to teach children programming. He talked about the fact that in universities, when the students start learning to program, they are often introduced to the features of the language, and first then a problem where those features were applicable. He thought that it was not an efficient way of doing it, as the students might ask why e.g. integers, floats, double and so on, was needed. The philosophy that they employ is the opposite, which focuses on presenting a problem where these features are needed to solve the problem. That way the children learn about the features knowing why, instead of ``just needing it to pass an exam''.

In the end, he stated that Coding Pirates was not actually for learning to be a great programmer, but making the attendants experienced in reading and understanding code and making them able to tweak code, but not creating code from scratch. The main focus of the events becomes game design rather than programming, when the attendants have a good enough knowledge base about programming. One of the observations that he mentioned was the fact that the attendants will start to split up into smaller teams and specialize in certain aspects of game creation e.g. one who draws the sprites, one who does the programming, one who comes up with the rules, etc. This makes them able to handle most aspects of game creation for a project internally in the team.