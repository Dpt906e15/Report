\section{DrRacket}
\label{sec:drracket}
DrRacket is an environment used to learn to write Racket code.
Racket is a functional programming language, and therefore DrRacket is our representative for an educational programming environment for the functional paradigm.
Worth noting is that all of us have learned to program in an imperative paradigm first and do not have much experience working with functional languages.
This will likely impact our code examples and opinion on the criteria evaluation.

\subsection{Fibonacci}
The first code example we will show in DrRacket is the Fibonacci implementation:

\begin{lstlisting}
(define (fibonacci n)
  (if (or (= n 0) (= n 1))
      1
      (+ (fibonacci (- n 1)) (fibonacci (- n 2)))))
\end{lstlisting}

This is a simple recursive implementation of Fibonacci with no memory optimizations.
Recursion is second nature to functional programming languages, so this is an intuitive implementation.
The function takes in a number to find the Fibonacci number of and then calls itself recursively on the two preceding numbers to get the two numbers it needs to sum up.
Eventually a trivial case of the called number being one or zero in which case it simply returns one.

\subsection{Cups and Ball}
The next code example is the cups and ball example.
This example is intuitively solved in an object-oriented way and since Racket has objects and classes, we do it like that.
We define the class like so:

\begin{lstlisting}
(define Cup%
  (class object%
    (define holdsBall 0)

    (super-new)
    
    (define/public (AddBall)
      (set! holdsBall 1)
      )
    
    (define/public (HasBall)
      holdsBall)
))
\end{lstlisting}

Each object of the class cup has a variable \lstinline!holdsBall!, which is used to store whether this cup has a ball, were 1 means it has a ball.
They also have two functions: \lstinline!AddBall! which sets \lstinline!holdsBall! to 1, and \lstinline!HasBall! which returns \lstinline!holdsBall!.
The code then creates a list of 15 balls and calls \lstinline!AddBall! on one of them chosen randomly.
The main game loop is facilitated with a recursive function \lstinline!AskUser!:

\begin{lstlisting}
(define (AskUser)
   (define guess (read))
  (if (= (send (list-ref cups (- guess 1)) HasBall) 1)
      (println "Congratulation, you found it!")
     (begin
       (println "Miss! Try again, pick a cup between 1 and 15: ") (AskUser))))
\end{lstlisting}

Here the user is prompted for a number between 1 and 15.
The cup on that position on the list then has its \lstinline!HasBall! function called.
If it is 1 the user is congratulated and the game ends, otherwise the user is prompted to guess again and the \lstinline!AskUser! function is called to repeat the cycle.

\subsection{Hangman}


\subsection{Criteria Evaluation}