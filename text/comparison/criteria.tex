\chapter{Criteria}
\label{chap:criteria}

Before a language analysis can take place, a set of criteria are needed. These criteria represent the foundation for the evaluation, and are chosen from a mixture of subjective viewpoints.

This chapter will present the criteria for the evaluation. Several criteria have been considered, where most are taken from known criteria for comparison \cite{design_criteria1} \cite{design_criteria2}, and others are found through discussion of the research questions.

\begin{description}[style=nextline]
\item[Readability] Readability is how easy code is to read and understand. Understanding can both be in sense of understanding what a specific procedure does but also how the procedure fits in with the rest of the code. Readable code gives a greater understanding of the semantics as well as the nature of the code. For this project, readability is a vital criteria, as interest often follows understanding. Readability is hard to measure, as it is a subjective matter, mostly depending on the person writing code.
\item[Writability] Writability is the ability to translate thoughts into code. It describes the expressivity of the code and the ease of writing, in a combination of quality and quantity. Writability is often found in the level of abstraction, and in the simplicity of the syntax.
\item[Observability] Observability is the level of feedback gained for a better understanding of how the code affects the project. It is to what extend one can observe how changes affect the result. It is found in the level of feedback, and the informative value it provides.
\item[Trialability] Trialability is the level of possibility for trial and error through coding. This feature is seen in the level of feedback when an error occurs, how often feedback is given, and how easily a programmer can recover from a mistake.
\item[Learnability] Learnability is the ease of learning the language. It defines how long it takes to learn the language and be proficient in its use.
\item[Reusability] The level of possibility for reusing code, through abstraction. Examples can be found in the use of functions, procedures and classes.
\item[Pedagogic Value] A programming language in itself can be easy to learn, but if it doesn't help the programmer in learning the basic concepts of programming, then there is no pedagogic value. This means the programming language should support the general programming concepts that are typical for common languages and coding conventions.
\item[Environment] The development environment plays an important part for novices, and it should provide help for the programmer in a simple, clear and manageable way.
\item[Documentation] The amount of documentation, as well as the informative value of this, is important for a novice, as a help for solving problems they cannot solve themselves.
\item[Uniformity] The consistency of appearance and behavior of language constructs. If the code does not look like any conventional language, the programmer will possibly have trouble moving on from this language. Constructs should be similar to commonly known syntaxes in the given paradigm, or preferably written in the exact same way.
\item[Miscellaneous] There is a possibility that other points of interest will be discovered during the comparison. These will be described in this category.
\end{description}

\section{Criteria Evaluation}
As the criteria are hard to measure in an objective way, the evaluation will be conducted through a subjective discussion. Each of the environments are meant for different purposes, as they are based on different paradigms. Taking this into account, we will create solutions to problems that are fit for each of the environments, and then compare these solutions in how all the environments handle and solve this problem, based on the criteria.