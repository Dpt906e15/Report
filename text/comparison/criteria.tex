\chapter{Criteria}
\label{chap:criteria}
\todo{new introduction to this chapter}
To measure the differences between languages with focus on novice learning, a set of criteria have been set up. These criteria are the base from the evaluation, and are a mixture of measurable and subjective criteria.

This chapter will present the criteria for the evaluation, the reason for the criteria being chosen, and how to measure them. Several criteria have been considered, where most are taken from known criteria for comparison \cite{design_criteria1} \cite{design_criteria2}, and others are found through discussion of the research questions.

\begin{description}[style=nextline]
\item[Readability] Readability is the expressibility of the language. Readable code gives a greater understanding of the semantics as well of the nature of the code. For this project, readability is a vital criteria, as interest often follows understanding. Since high readability produces greater understanding, readability is chosen as a criteria for comparison. Readability is hard to measure, as it is a subjective matter, mostly depending on the person writing code. The measurement of this quality will therefore based on using code conventions and skilful coding.
\item[Writability] Writability is the ability to translate thoughts into code. It describes the expressivity of the code and the ease of writing, in a combination of quality and quantity.Writability can be measured in the level of abstraction. This can be done through lines of code as well as looking into the different language constructs that support abstraction.
\item[Observability] Observability is the level of feedback gained for a better understanding of the input. It is to what extend you can observe reactions to what you make. To measure the observability is lo look at this level of feedback.
\item[Trialability] The level of possibility of trial and error through coding is measured through trialability. This feature is measured by the level of feedback when an error occurs, how often feedback is given, and a discussion on how easily a novice programmer can recover from a mistake.
\item[Learnability] As a language is learned, there are helping and hindering factors. These are measured through learnability, which is done by measuring the cost of learning the language and its environment.
\item[Reusability] The level of possibility for reusing code, through abstraction. This is measured both in quantity and quality, depending on the abstraction level and layer depth.
\item[Pedagogic Value] A programming language in itself can be easy to learn, but if it doesn't help the programmer in learning the basic concepts of programming, then there is no pedagogic value. This means the programming language should support the general programming concepts that are typical for common languages and coding conventions.
\item[Environment] The development environment plays an important part for novices, and it should provide help for the programmer in a simple, clear and manageable way. The evaluation of the environment will be done according to how interactive systems are evaluated by David Benyon \cite[p. 225-250]{design_book}.
\item[Documentation] The amount of documentation, as well as the informative value of this, is important for a novice, as a help for solving problems they cannot solve themselves.
\item[Security] The level at which violations of definitions are caught by the compiler. \todo{do we need this one?}
\item[Uniformity] The consistency of appearance and behavior of language constructs. If the code doesn't look like any conventional language, the programmer will not learn the general approach to programming, and will have trouble moving on from this language. This is measured in terms of constructs being similar to known syntaxes, or preferably written in the exact same way.
\end{description}