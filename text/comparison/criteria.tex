\chapter{Criteria}
\label{chap:criteria}
To compare languages for their usability and suitability for teaching, a set of measurable criteria should be found. These criteria should help making the comparison more than a subjective discussion, and also give a greater insight for a discussion in that manner.

This chapter will present the criteria for the evaluation, the reason for the criteria being chosen, and how to measure them. Several criteria have been considered, where most are taken from known criteria for comparison \cite{design_criteria1} \cite{design_criteria2}, and others are found through discussion of the research questions.

\section{Readability}
Readability is the expressibility of the language. Readable code gives a greater understanding of the semantics as well of the nature of the code. For this project, readability is a vital criteria, as interest often follows understanding. Since high readability produces greater understanding, readability is chosen as a criteria for comparison.

\todo{How to measure this for experiment.}

\section{Writability}
Writability is the ability to translate thoughts into code.

\section{Observability}


\section{Trialability}


\section{Learnability}


\section{Reusability}


\section{Pedagogic Value}


\section{Environment}


\section{Documentation}


\section{Security}


\section{Uniformity}

