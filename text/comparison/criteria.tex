\chapter{Criteria}
\label{chap:criteria}
\todo{new introduction to this chapter}

This chapter will present the criteria for the evaluation, the reason for the criteria being chosen, and how to measure them. Several criteria have been considered, where most are taken from known criteria for comparison \cite{design_criteria1} \cite{design_criteria2}, and others are found through discussion of the research questions.

\section{Readability}
Readability is the expressibility of the language. Readable code gives a greater understanding of the semantics as well of the nature of the code. For this project, readability is a vital criteria, as interest often follows understanding. Since high readability produces greater understanding, readability is chosen as a criteria for comparison.

Readability is hard to measure, as it is a subjective matter, mostly depending on the person writing code. The measurement of this quality will therefore based on using code conventions and skilful coding.

\section{Writability}
Writability is the ability to translate thoughts into code. It describes the expressivity of the code and the ease of writing, in a combination of quality and quantity.

Lines of code, abstraction.

\section{Observability}


\section{Trialability}


\section{Learnability}


\section{Reusability}


\section{Pedagogic Value}


\section{Environment}


\section{Documentation}


\section{Security}


\section{Uniformity}

