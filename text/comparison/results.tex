\chapter{Results}
\label{chap:results}

The three languages have been analysed through a set of subjective criteria, and through this it shows that Scratch is the language and environment best suited for novice programming. Its level of intuition and well-structured environment makes it possible for a novice to understand the concepts without the need of external resources. Unfortunately, this fact does not persist in BlueJ and DrRacket. Scratch is extremely easy to read, as its syntax for statements resembles natural language. The feedback level gives possibility for great observability and trialability, which leads to a more steady learning curve. Many of these factors contribute to the learnability of Scratch, which therefore also makes Scratch the language best suited for novices among the languages compared. BlueJ is ahead in both usability and uniformity, which results in a language common to other languages in its paradigm, and with a great expressibility. Its graphical representation of classes offer a great overview of the class hierarchy, which is a great help for novices. BlueJ unfortunately isn't as intuitive for novices without help, such as a teacher or tutorial. DrRacket had its merits in different areas, where writability was one of those of higher value. When understood, the language converts thoughts to code in a fast paced manner, but can be extremely hard to understand without help from outside the environment.

An interesting evaluation would be to compare the languages through the criteria used to develop Grace\cite{grace}, as this is meant to be an educational language. There are three main criteria used to develop Grace, being object construction, type checking and language levels. 

As object construction is a feature only available to object-oriented languages, this can be evaluated through the simplicity of implementing core features for each paradigm. Scratch is an imperative language, with a hint of procedural programming, and the nature of the paradigm is to declare statements. This objective is filled by using blocks for each type of statement. This clearly defines what the statement does, and makes a clear separation in the different statements. BlueJ is object-oriented, as it is an environment for Java programming. The main feature in this paradigm is creating and using classes and objects. BlueJ offers a graphical interface, where classes can be created and related. This offers a great help for novices, to get a greater overview of the structure of the code. DrRacket is for Racket, which is a functional programming language. The main feature in functional programming is expressions. Racket in itself consists of expressions, and it is easy to create and use expressions. DrRacket provides a text editor for coding, but not much more help is given in this criteria.

Type checking is a factor which depends on whether the language is strongly or weakly typed. Scratch has only a very small set of data types, and the only collection type is the list. This makes the language very easy to control, and the reliability of the language is therefore of high level. The language itself handles all type checking, and the user is not needed to understand types in depth. BlueJ is based on Java, and it has the possibility of choosing. As a coding convention, the type is normally strongly typed, but keywords such as \emph{var} make it possible for weak typing, as well a more dynamic typing. Unless used correctly, though, the program may not compile, as a perfect reliability for a multi-purpose language such as Java is near impossible. DrRacket is a weakly typed language, which makes it possible to make some fairly abstract expressions, in the form of functions. For an advanced programmer, this leads to a wide list of possibilities. As a novice, however, the possibility of using the expressions incorrect is very large.

Language levels is actually an idea for Grace inspired by DrRacket\cite{languagesAsLibraries}. DrRacket has a feature of selecting languages depending on experience, which is a great asset for the language to be fitting all experience levels. None of the other languages have this, where BlueJ only provides an easier approach to a multi-purpose language, and Scratch is almost only fit for novice programming. Although the choice of experience level provided in DrRacket helps throughout the tutorial in understanding the language, it still does not help much without external resources.

The languages and environments almost fill the requirements set by the people behind Grace. Scratch and BlueJ do not have different language levels. Scratch is focused on novice programmers, whereas BlueJ has chosen to handle the problem through an environment instead. This problem is of no greater concern to this project, as the focus is only on novice programming. With that said, the conclusion might also suggest that Scratch works better for novices than the other languages and environments, since it focuses only on this aspect. With this short comparison in mind, Scratch still comes out best of the three compared tools.