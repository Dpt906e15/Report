\chapter{Conclusion}
\label{chap:conclusion}

Through this paper, we have analysed the field of novice programming and the problems regarding learning programming. We have looked through several languages for novices, an discussed their merits and shortcomings. There is a huge difference in visual and text based programming, which is addressed, where a conclusion is found in the targeting groups. Visual programming works best for novice programmers, whereas text based programming is better for a more experienced programmer. We have analysed the field of teaching, and concluded that the methods for teaching varies from country to country. The U.K. seems to have a good grasp on novice programming, where Denmark have some trouble, as the teaching is done by possibly unqualified teachers.

We chose to carry on the analysis through three novice languages and environments, being Scratch, BlueJ and DrScratch, with a language comparison. This was carried out as an expert evaluation by the authors. We analysed each language by a given set of criteria, which were criteria for measuring programming languages in general, and for measuring the interface design. Some sample pieces of code were made for comparison, which were made to match each of the paradigms representing the languages. After discussing each criteria for each language, with examples through the sample code, we compared the languages in each criteria. We concluded that a great difference lies in the targeting age. Scratch can work as a great introduction to programming down to the age of a primal school pupils. BlueJ serves as a great introduction to general purpose programming, but requires more effort and concentration. DrRacket can be hard, and must be taught by a teacher with great understanding in the underlying concept of functional programming. Being biased by the imperative structure, the declarative structure of functional programming was hard to put our minds into. Therefore, understandings of the ease of learning such a paradigm can be hard, and our understanding is therefore up to discussion.