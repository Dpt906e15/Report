\section{Readability}
\label{sec:readability}
All the three environments we have selected have a varying degree of readability. This is directly tied to the target groups each one tries to address and their respective age. For this reason, we consider Scratch to have the highest readability of the three since it is considered the most verbose and it is the only environment of the three which supports a fully visual programming language. That helps immensely with how programs are structured and represented through the use of simple, modular blocks and how easy is to understand different parts of the code. Keywords in the language are named in such a way that are more descriptive of the action they convey, adding to the verbosity of the language, and more intuitive in terms of understanding compared to the traditional approach used both by BlueJ and DrRacket. 

On second place we have BlueJ which is mainly used in CS1 courses or pre-college courses, involving people of that respected age. This naturally means that object-orientation practices are used for creating programs and there is a heavy emphasis on classes and objects which are modelled after their real world counterparts and highschoolers and students have easier time understanding them. Although not anywhere near the level of Scratch, BlueJ has a visual representation of classes and instances of classes and objects which gives a better overview and understanding of the hierarchical structure of programs than traditional IDEs used with Java. 

We place DrRacket last in terms of readability since its entirely textual approach with combination of unconventional use of mathematical operations, brackets and lack of infix operators makes it the hardest to understand environment for novice programmers of varying age, in terms of its underlying programming language adhering to the functional paradigm.

