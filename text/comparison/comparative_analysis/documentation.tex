\section{documentation}
\label{sec:documentation}
Each of the language environments have different ways of providing documentation. BlueJ has an advantage in being an environment for programming Java, which leads to a lot of documentation and helping forums. Furthermore, it has documentation on how to use the environment itself. Scratch as a language is very dependent on its environment, as it is a visual language, which leads to a mixture of environment and programming documentation. DrRacket has little documentation on how to use the environment as a whole, but more about how to use it to code. This means most of the documentation is about Racket as a language.

There is a possibility of finding online tutorials for all the languages. Scratch is very educationally friendly in that area, as it provides a tutorial as a part of the programming window. DrRacket also offers a very thorough guide and reference work on their homepage, and BlueJ offers a paper, serving as a tutorial and guide for the environment. The documentation for BlueJ is unfortunately not very novice friendly, as it assumes the programmer is known in Java.