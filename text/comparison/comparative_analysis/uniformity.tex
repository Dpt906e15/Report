\section{Uniformity}
\label{sec:uniformity}

The three languages have little in common overall, which means a lower rating of internal uniformity. As BlueJ is an environment of a well known programming language, its uniformity is unquestionable when it comes to object-oriented programming. Scratch is very simplistic, and its basic constructs are similar to those in widely used languages, although they are differently named. It also has an advantage for novices, as the declarations often don't require knowledge of coding conventions for that type of expression. As an example, the block setting the value of a variable shows ``set [NAME] to [VAL]'', instead of the traditional "[NAME] = [VAL]''. In Scratch, [NAME] is even a drop-down list of possible variables, and [VAL] can both be written directly, or be another variable block. DrRacket is very different from well known languages, which leads to a low score in uniformity. The lack of infix operators and other convention in parentheses makes it hard to transition from this language to other languages.