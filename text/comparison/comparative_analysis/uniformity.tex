\section{Uniformity}
\label{sec:uniformity}

The three languages have little in common overall, which means a lower rating of internal uniformity. As BlueJ is an environment of a well known programming language, its uniformity is unquestionable when it comes to object-oriented programming. Scratch is very simplistic, and its basic constructs are similar to those in widely used languages, although they are differently named. It also has an advantage for novices, as the declarations often don't require knowledge of coding conventions for that type of expression. As an example, the block setting the value of a variable shows ``set [NAME] to [VAL]'', instead of the traditional \lstinline![NAME] = [VAL]!. In Scratch, \lstinline![NAME]! is even a drop-down list of possible variables, and \lstinline![VAL]! can both be written directly, or be a variable block. DrRacket, being an environment for a functional programming language, has a very different syntax to other paradigms. The lack of infix operators and other convention in parentheses makes it hard to transition from this language to other paradigms. This fact therefore also applies the other way around. On the other hand, Racket is built on a well known language, Scheme, and the transition to other languages in the same paradigm is easier. As even the operators are functions, it gives a great understanding to the nature of lambda functions.