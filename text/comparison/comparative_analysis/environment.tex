\section{Environment}
\label{sec:environment}
As scratch is a visual programming language, its environment differs greatly from both BlueJ and DrRacket. With this said, BlueJ also offers a visual representation of the structure. DrRacket has no graphical advantages that would help novices in learning to code, other than libraries that provide high order functions for creating shapes as an output. This results in Scratch and BlueJ being better for a younger targeting group. As an addition to this, Scratch also functions as a small game engine, with a lot of visual representations of its mechanics and output. This, along with the playful nature of the environment, makes Scratch the environment of the three best fitting children in primary school.

BlueJ is used for programming Java, and its environment is excellent in learning novices the concepts of objects and classes. It provides a lot of helping functionality, which is not found in other development environments for the language. Scratch is good for teaching younger novices the basic concepts of imperative coding. DrRacket helps in understanding declarative programming, which is much different from most well-known paradigms. They each have their merits, but as an educational tool for teaching novices, Scratch has the most helpful environment.