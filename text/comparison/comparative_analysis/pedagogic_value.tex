\section{Pedagogic Value}
\label{sec:pedagogic_value}
Each of the languages/environments provides pedagogic values for different aspects of programming. Scratch has pedagogic value in the sense that knowledge about basic low level concepts is easily gained. It has a possible shortcoming in a user transitioning to a text-based programming language, as some of the features that they are used to are missing. BlueJ does not provide a lot regarding basic concepts of programming, but provides a lot of tools to get an understanding of how classes and objects works. DrRacket provides pedagogic value in the sense that it can provide an understanding of the underlying functionalities in programming languages. As an example, in a lot of programming languages, the user will use the expression $2+2$ to get the value $4$, possibly assuming that it ``just works'', where DrRacket shows that using the $+$ operator is a function call in itself. DrRacket also encourages learning and using recursions, as it is the intuitive way of working with it compared to loops. 