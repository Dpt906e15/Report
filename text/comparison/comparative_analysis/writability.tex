\section{Writeability}
\label{sec:writability}
Similarly to how the three platforms were evaluated according to their readability, we consider Scratch the environment with the highest writeability, followed by BlueJ and lastly - DrRacket. Programs in Scratch are not only easy to understand but also relatively easy to make. It is actually the only one from the three environments which does not require any programming knowledge but rather intuition, intended results and trial-and-error since it is specifically designed for young kids. Compared to BlueJ and DrRacket which have a compilation step, Scratch extends its writeability through its reactive nature, giving immediate results from an executed code and faster time in correcting mistakes. Keywords in the language are named in such a way that are more descriptive of the action they convey and more intuitive in terms of understanding compared to the traditional approach used both by BlueJ and DrRacket.

BlueJ takes the second place since it does not provide much more beyond the standard in terms of writability established by Java, with the exception of the option to colour code, where different parts of the code take whatever colours the user specifies, taking into account colour blind people as well.

DrRacket takes the last place since the style of programming defined by the functional paradigm seems the least intuitive to work with. Additionally, as already mentioned in \ref{subsec:criteval}, this is compounded by the fact that the language lacks infix functions, thus common mathematical operations have different syntactical structure.