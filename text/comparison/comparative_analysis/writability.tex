\section{Writeability}
\label{sec:writability}

 Programs in Scratch are not only easy to understand but also relatively easy to make. It is actually the only one from the three environments which does not require any programming knowledge but rather intuition, intended results and trial-and-error since it is specifically designed for young kids. Compared to BlueJ and DrRacket which have a compilation step, Scratch extends its writeability through its reactive nature. Furthermore, great efforts were made during the development of Scratch so syntactical errors could be avoided for the majority of the cases and in terms of semantics, there is intentionally a limited set of possible data types and blocks which could be given as arguments at any given time which reinforces the intuitiveness of the language. However, given the visual programming nature of Scratch, it is generally slower to achieve a given result compared to BlueJ and DrRacket. 	 

BlueJ does not provide much more beyond the standard in terms of writability established by Java, with the exception of the option to colour code, where different parts of the code take whatever colours the user specifies, taking into account colour blind people as well. However, Java makes use of constructs with widely known structure and semantics, similar to many popular languages, making it easier to become accustomed with the environment.

DrRacket has a style of programming defined by the functional paradigm seems the least intuitive to work with when novice programmers are considered. Additionally, as already mentioned in \secref{subsec:criteval}, this is compounded by the fact that the language lacks infix functions, thus common mathematical operations have different syntactical structure. However, it generally takes less number of lines to implement a given task compared to BlueJ and Scrach's more object-oriented approach, which directly affects the writability of the environment.