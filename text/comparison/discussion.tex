\chapter{Discussion}
\label{chap:discussion}

\section{Process}

During the initial stage of our project we spend a great deal of time and effort on narrowing down the specific problem we wanted to address in the research area of our choice. The field of programming is quite big and varied and it encompasses many research areas and avenues one can explore. In order to address  the raising demand for programmers worldwide, educational institutions spend more and more time and resources on properly educating people for such a monumental task. However, the is not a well established way which fits all of achieving this task, especially for people in the early stages of their lives. Initially, we decided on exploring how to bridge the gap between educational and professional programming languages. But through our research and analysis of the existing research domain we later focused on identifying what are the difficulties which young people face when they start learning programming and what are the most common and severe mistakes they usually make. We tried to cover as much as we could from the research area by analysing both text-based and visual-based programming languages. Additionally, language comparison was conducted on well established programming paradigms with common mistakes and pitfalls which might come with them and on three specific programming environments  with their underlying languages.
\subsection{Difficulties}

There were several difficulties which we faced while working with our research question. First, we had a tough choice identifying which might be the most severe problems pupils face when given programming tasks to solve. The research on that specific area is highly opinionated and there is not a single right and perfectly accurate answer how to achieve that. Therefore, we considered and selected the problems we found most prevalent in the majority of the research materials we covered and through our own experiences with the languages and environments. Second, we spend considerable time on deciding which are the most relevant programming paradigms in the context of teaching programming and which educational languages are best fit for representing those paradigms, reflected by their usage by novices. Last but not least, we had to familiarize ourselves with each one of the three programming environments we have chosen in order to acquire a better overall understanding of their inner workings and to be able to give our objective opinion as part of the analysis. Given that we all a background in imperative programming, we found DrRacket the hardest to understand because of its underlying functional language.
\subsection{Scratch}
We consider Scratch as a great choice for teaching programming to novices given its popularity and ease of use. It provides a fast way to implement different ideas without the need to have knowledge about the entirety of the platform. However, we are still conscious about its shortcomings and how novices can quickly outgrow the platform. Although the verbosity of the language is one of its main strengths, it is also one of its weaknesses since the code build mostly by blocks tends to get very big for medium-sized projects, which might significantly reduce its readability, and impractical for larger ones. We are aware that this is an intentional design decision made by the Scratch team, since the purpose of the language is to teach the basic concepts of programming rather than be a fully-fledged programming tool for building programs. 
\subsection{Our thoughts}
\todo{Maybe we have to discuss this first?}
\subsection{Documentation}
\todo{No idea what should be here}
\subsection{Missing evidence}
\todo{No idea what should be here}
\subsection{Danish education}
Digital literacy is an essential component of a modern education, which can directly affect a particular country's economy. Currently, there is a growing shortage of people skilled in ICT and according to (\todo{there should be a ref here}), an estimate is that there will be 900,000 vacant jobs in Europe's ICT Secton by 2020. This is one of the main reasons why the introduction of programming as a mandatory course became a reality in Denmark in 2014, following the trend established in several European countries, the first of which was England. Based on their age, children are getting lessons on algorithms, coding in languages such as Java, and debugging programs. However, at this point there are no dedicated teachers for teaching programming, but rather that role is filled in by the physics teacher in schools. This might mean that the learning experience which is provided might not be on a satisfactory level but there is still no sufficient data to support that claim.

\section{Coding Pirates}
\label{sec:coding_pirates}

