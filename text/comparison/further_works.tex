\chapter{Further Works}
\label{chap:further_works}
Following the conclusion of Staking Claims\todo{ref needed} the programming language community, including us, need to provide some scientific data behind any claim made on the presentation of the language.
The common suggestion is to use the techniques from social sciences to gather data, but these techniques often require a number of participants in the thousands, to make the data quantitative enough.
This is usually beyond the resource available to the language developers, and far beyond our resources.
At the same time there has been a large push to make programming more accessible to the general public, which is especially evident with Scratch, which is built to be intuitive enough to be used without external assistance.
This leads us to the idea of using the usability test from human-computer interaction on programming language environments.
These tests are used to see where a program has problems with the affordance that allows users to easily interact with the system.
Since programming language environment can be seen as a program to allow the user to interact more easily with the language, this means the tests are applicable on them.
Furthermore usability tests only require a few participants to say a lot about the program making it a lot more accessible to developers, who in fact are already frequently using them to improve their programs.
These tests could both be used to analyse the existing languages as well as any new language we might develop.

