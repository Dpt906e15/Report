\chapter{Introduction}
\label{chap:introduction}
Traditionally, programming has been seen as a specialized skill, only relevant for people who work on developing new software.
While several attempts have been made at making languages more accessible to everyone \cite{turtle_func, smallbasic_faq}, they ultimately failed to get enough widespread use to change this perception.
As a result, programming education has been targeted towards college or university students and often had a focus on teaching the student to use general purpose programming languages like Java\cite{tiobe}.
The goal here is to create competent software engineers with the ability to work with powerful tools on complicated software.

In recent years, programming is starting to be seen as an essential skill for living in our modern digital society.
This has lead to many countries making programming a mandatory subject in primary schools. In this setting, the education is more focused on giving children an idea of what it is like to work on software and to teach more generic skills, such as problem solving and collaborative communication. This leads to this education usually being given in visual educational languages like Scratch \cite{MaloneyResnick10} for their intuitiveness and simplicity. This means that different levels of education differ in what they teach.
First, there is a difference in programming languages, as to between being taught an educational language and a general purpose language. While the educational language has the advantage of being intuitive, it does not have the expressiveness and robustness for large projects that general purpose languages have.
Second, the computer science education needs to cover a lot of topics to give a sufficient understanding to work professionally with software, where the kids education leaves out a lot of the topics, although this may vary between teachers.
Using the list of topics from "What do Teachers Teach in Introductory Programming?"\cite{WhatDoTeach06} as a reference, we can for example say that topics like algorithm design and debugging are likely to be taught in the kids education. Meanwhile, topics like algorithm efficiency, pointers and object oriented programming are usually left out.

Kids now being taught programming in primary school is great for the general digital literacy of people, but the education can not teach everything necessary to do professional programming.
We want to understand the educational value of teaching novice programmers programming in primary school. To do this, we first need to understand how and what novices are taught. The report therefore aims to understand the areas where novices in primary school have problems as well as the tools used for teaching programming.

\section{Initial Questions}
Today, more and more countries are implementing computational thinking as a part of the primary school curriculum. We want to get an understanding of educational programming. To do this, we want to analyse the literature and tools used for this. The difference in languages and environments must also be addressed. As there is a big difference in visual and text-based programming, the difference between these must be analysed. Different paradigms in programming also have different approaches to problem solving. This might lead to the fact that novices have to learn aspects that help them code in the specific paradigm. These factors must be explored as well.

Through this, we have the following initial questions:

\begin{itemize}
  \item Which tools are being used by novices regarding their first contact with programming?
  \begin{itemize}
    \item How can these tools be compared?
    \item What do the tools lack?
    \item Which programming paradigms are being taught?
    \begin{itemize}
      \item What are the benefits of each paradigm?
    \end{itemize}
  \end{itemize}
  \item What are the difficulties in learning programming?

  \item What are the differences of existing educational programming languages?
  \item Is it difficult to go from visual programming to text-based programming?
\end{itemize}

\section{Report Structure}
In the first part of the report we analyse the area as a whole. We first present some related work in \chapref{chap:related_work}. Next, in \chapref{chap:error-prone_areas_for_novices}, we discuss the various problems that novices are faced with when learning to program. We then present different educational languages and environments, and make an initial comparison between visual and textual languages in\chapref{chap:languages_and_tools}. In \chapref{chap:teaching} we analyse what is done to implement teaching into the school curriculum in a couple of different countries.

In the second part of the report, we give a language comparison, done as a subjective evaluation through discussion by the authors. We state the criteria evaluated in the comparison in \chapref{chap:criteria}. We analyse the chosen languages in \chapref{chap:language_analysis}. The comparative analysis itself is provided in \chapref{}. The results of the analysis is then discussed in \chapref{chap:results}.

In the final part of the report, we have a conclusion in \chapref{chap:conclusion}, and a discussion of the project and ideas for further works in \chapref{chap:discussion}.
%We want to work with teaching the skills for professional programming using the already taught knowledge from primary school.
%To do this we first need to establish what level people have after working in the educational language and what knowledge is needed to work professionally with programming.
%This paper aims to do this as well as suggesting some ideas for ways to bridge that gap.

%I'm not the only one worried that programming is hard and should be made easier in general
%http://www.technologyreview.com/view/429438/dear-everyone-teaching-programming-youre-doing-it-wrong/

%http://worrydream.com/LearnableProgramming/

%other links:
%http://dl.acm.org.zorac.aub.aau.dk/citation.cfm?id=362052.362053&coll=DL&dl=ACM&CFID=551259560&CFTOKEN=42841981

%http://www.learntomod.com/

%http://dl.acm.org.zorac.aub.aau.dk/citation.cfm?id=2594413.2591203&coll=DL&dl=ACM&CFID=551259560&CFTOKEN=42841981
