\chapter{Introduction}
\label{chap:introduction}
Traditionally, programming has been seen as a specialized skill, only relevant for people who work on developing new software.
While several attempts are made at making languages more accessible to everyone \todo{reference logo and probably some more}, they ultimately failed to get enough widespread use to change this perception.
As a result, programming education has been targeted towards college or university students and often had a focus on teaching the student to use professional programming languages like Java\todo{can probably use source about the widespread use of Java here}.
The goal here is to create competent software engineers with the ability to work with powerful tools on complicated software.

In recent years, programming is starting to be seen as an essential skill for living in our modern digital society.
This has lead to many countries making programming a mandatory subject in primary schools. In this setting, the education is more focused on giving children an idea of what it is like to work on software and to teach more generic skills, such as problem solving and collaborative communication. This leads to this education usually being given in visual educational languages like Scratch \cite{MaloneyResnick10} for their intuitiveness and simplicity. This means that different levels of education differ in what they teach.
First, there is a difference in programming languages, as to between being taught an educational language and a professional language. While the educational language has the advantage of being intuitive, it does not have the expressiveness and robustness for large projects that professional languages have.
Second, the computer science education needs to cover a lot of topics to give a sufficient understanding to work professionally with software, where the kids education leaves out a lot of the topics, although this may vary between teachers.
Using the list of topics from "What do Teachers Teach in Introductory Programming?"\cite{WhatDoTeach06} as a reference, we can for example say that topics like algorithm design and debugging are likely to be taught in the kids education. Meanwhile, topics like algorithm efficiency, pointers and object oriented programming are usually left out.

Kids now being taught programming in primary school is great for the general digital literacy of people, but the education can not teach everything necessary to do professional programming.
We want to understand the educational value of teaching novice programmers programming in primary school. To do this, we first need to understand how and what novices are taught. The paper therefore aims to understand the areas where novices in primary school have problems as well as the teaching methods used in the education.

\section{Initial Questions}
We need to know what and how kids are being taught. These methods must be analysed, discussed and compared. As the methods not necessarily cover all the problems in teaching, we also must address the difficulties of learning programming as a whole. The difference in languages and environments must also be addressed. As we already know there is a big difference in visual and text-based programming, the difference between these must be analysed. Different paradigms in programming also have different ways of expressing themselves. This might lead to the fact that novices have to learn aspects that help them code in the specific paradigm. These factors must be explored as well.

Through this, we have the following research questions:

\begin{itemize}
  \item What is being taught to the children as their first contact with programming?
  \begin{itemize}
    \item How can these methods be compared?
    \item What do the methods lack?
    \item From the children's point of view, what is the motivation for learning programming?
    \item Which programming paradigms are being taught?
    \begin{itemize}
      \item What are the benefits of each paradigm?
    \end{itemize}
  \end{itemize}
  \item What are the difficulties in learning programming?

  \item What is the difference between educational programming languages and professional programming languages?
  \begin{itemize}
    \item What is the definition of these?
  \end{itemize}
  \item Is it difficult to go from visual programming to text-based programming and vice versa?
  \item How does knowledge of different topics, such as math or object-oriented design, facilitate the introduction to programming?
\end{itemize}

\section{Report Structure}
The first thing we will address is an analysis of the area as a whole. In \chapref{chap:error-prone_areas_for_novices}, we discuss the various problems that novices are faced with when learning to program. We will then continue with an analysis of different educational languages in \chapref{chap:languages_and_tools}. After that, we provide an analysis on the teaching as it is today in \chapref{chap:teaching}.

After an analysis of the area, we will give a language comparison, done as an expert evaluation. The introduction to why and how this is done will be provided in \chapref{chap:preliminaries}. We will state the criteria evaluated in the comparison in \chapref{chap:criteria}. The comparative analysis itself will be provided in \chapref{chap:language_analysis}. A conclusion on the analysis will be given in \chapref{chap:conclusion}, which also gives a discussion on the matter and ideas for further works.
%We want to work with teaching the skills for professional programming using the already taught knowledge from primary school.
%To do this we first need to establish what level people have after working in the educational language and what knowledge is needed to work professionally with programming.
%This paper aims to do this as well as suggesting some ideas for ways to bridge that gap.

%I'm not the only one worried that programming is hard and should be made easier in general
%http://www.technologyreview.com/view/429438/dear-everyone-teaching-programming-youre-doing-it-wrong/

%http://worrydream.com/LearnableProgramming/

%other links:
%http://dl.acm.org.zorac.aub.aau.dk/citation.cfm?id=362052.362053&coll=DL&dl=ACM&CFID=551259560&CFTOKEN=42841981

%http://www.learntomod.com/

%http://dl.acm.org.zorac.aub.aau.dk/citation.cfm?id=2594413.2591203&coll=DL&dl=ACM&CFID=551259560&CFTOKEN=42841981
