\chapter{Introduction}
\label{chap:introduction}
\todo{This was initially a chapter in the report, but as it came to be somewhat a presentation of the motivation for the project, we moved it to the introduction. It is not finished as it is, but the general idea is there.}
Traditionally programming have been seen as an engineering discipline.
As a result programming education has been targeted towards college or university students and often had a focus on teaching the student to use professional programming languages like Java.
The goal here is to create competent software engineers with the ability to work with powerful tools on complicated software.

However in recent years programming is starting to be seen as an essential skill for living in our modern digital society.
This has lead to many countries making programming a mandatory subject in primary schools.
In this setting the education is more focused on giving children an idea of what it is like to work on software and to teach more generic skills like problem solving and collaborative communication. This leads to this education usually being given on visual educational languages like Scratch for their intuitiveness and simplicity.

Of course this means that the two educations differ in what they teach.
First the difference in programming languages between being taught an educational language and a professional language. While the educational language has the advantage of being intuitive, it does not have the expressiveness and robustness for large projects that professional languages have.
Second the computer science education needs to cover a lot of topics to give a sufficient understanding to work professionally with software, where the kids education leaves out a lot of the topics, though which ones may vary between teachers.
Using the list of topics from "What do Teachers Teach in Introductory Programming?"[bottom link] as a reference, we can for example say that topics like algorithm design and debugging are likely to be taught in the kids education. Meanwhile topics like algorithm efficiency, pointers and object oriented programming are usually left out.

Kids now being taught programming in primary school is great for the general digital literacy of people, but the education can not teach everything necessary to do professional programming.

\todo{The idea with this section is to discuss the differences between what is being taught as kids programming versus professional programming}
%I'm not the only one worried that programming is hard and should be made easier in general
%http://www.technologyreview.com/view/429438/dear-everyone-teaching-programming-youre-doing-it-wrong/

%http://worrydream.com/LearnableProgramming/

%other links:
%http://dl.acm.org.zorac.aub.aau.dk/citation.cfm?id=362052.362053&coll=DL&dl=ACM&CFID=551259560&CFTOKEN=42841981

%http://www.learntomod.com/

%http://dl.acm.org.zorac.aub.aau.dk/citation.cfm?id=2594413.2591203&coll=DL&dl=ACM&CFID=551259560&CFTOKEN=42841981

%What do teachers teach:
%http://dl.acm.org.zorac.aub.aau.dk/citation.cfm?id=1151588.1151593&coll=DL&dl=ACM&CFID=551259560&CFTOKEN=42841981