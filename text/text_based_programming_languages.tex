\section{Text Based Programming Languages}
Text based programming languages can be split into two categories; one is the text based educational programming languages, for novices and the other are the general purpose programming languages. Eventhough general purpose programming languages can be used to teach programming to novices, the two categories are distinguished in where the focus of the design has been. This chapter will explore the programming languages in both of these categories.

\subsection{Text Based Educational Programming Languages}
One of the first programming languages that added constructs for learning programming was LOGO. It did not have the constructs from the start, but after 12 seventh-grade students\footnote{From Muzzy Junior High School in Lexington, Massachuseets.} worked with LOGO for a year (1968-1969), Seymour Papert, one of the developers of LOGO, proposed the Turtle as a programming domain that could be interesting to people at all ages. He proposed it since the demonstration had confirmed that LOGO was a learnable programming language for novices, but he wanted the demonstration extended to lower grades, ultimately preschool children. Constructs for Turtles was then added to LOGO and has since been widely adopted in other programming languages such as SmallTalk and Pascal, and more recently Scratch.

A Turtle can be a visual element on a screen\footnote{In e.g. Scratch a Turtle can be any sprite chosen by the programmer.} or a physical robot. In LOGO the Turtle is controlled by a set of commands which are:
\begin{itemize}
\item FOWARD X, moves the Turtle X number of Turtle steps in a straight line
\item RIGHT X, turns the Turtle X number of degrees in a clockwise direction
\item LEFT X, turns the Turtle X number of degrees in a counter clockwise direction
\item PENDOWN, makes the Turtle draw
\item PENUP, makes the Turtle stop drawing
\end{itemize}
These commands make up the essence of Turtle programming and is also present in the other languages which has implemented Turtle programming. Some languages has expanded on these commands e.g. in Scratch, one can change the color, size and shade of the pen.

\todo{What is turtle programming supposed to teach?}\\
\todo{Other languages}\\
\todo{GRAIL}\\
\todo{transition to visual programming}\\

\subsection{General Purpose Programming Languages}\\
\todo{Why is it the goal to learn these languages?}\\
\todo{Why can't we stick with the educational ones?}\\
\todo{Why don't we begin with these?}
\todo{Tools for learning e.g. BlueJ}

%http://dl.acm.org/citation.cfm?id=1227003
%paper of physical programming