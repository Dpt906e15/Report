\chapter{Teachers and Programming}
\label{chap:teaching}
Around the world, learning computational thinking is starting to appear in the school curriculum for students in the age range of 5-16 years old, depending on the country. Because of this it is relevant to study which resources are available to the teachers who are going to teach the children. In this chapter we will describe the current state of the teachers and discuss possible problems regarding that state. The chapter will focus mainly on Denmark and the United Kingdom.

\section{Current State}
As of 2015 the danish government has added programming to the school curriculum, although this might be a wrong way of phrasing it as, according to the Danish Learning Portal\footnote{http://www.emu.dk} it is not only programming that should be taught but computational thinking in general. An example of this can be seen in this quote on how to incorporate ``programming'' into the mathematics course\footnote{This is translated from danish, so the wording might be different in the original text, but the meaning is the same}:
\begin{quote}
Programming activities can support that the students work with algorithms, meant as systematic descriptions of issues, solution strategies and events. A recipe is a good example of an algorithm. (1) Mix the dry ingredients together, (2) stir. (3) Add 2/3 of the water and stir. (4) If the dough is smooth, stir for 2 minutes. Else go to step (3) and add more water. Algorithmic thinking is about setting up and making machines execute such algorithms...\cite{dk_scratch}
\end{quote}

The problem with this addition to the curriculum is that the danish government expect the individual teacher to teach themselves the subject, to then teach the students. The government has allocated one billion Danish Crowns to educate current teachers in the new subjects added to the curriculum for 2015, but programming is only one subject among many. This means that it is uncertain how much of it goes to educating teachers in the ability to teach programming. One of the things the government has suggested is the programming environments which can be used, here among ``Scratch'', ``Tynker'' and websites like Code.org.

Another part that is uncertain is how the educations regarding teaching will implement the new curriculum. It is the individual educational institution's job to make certain that the teachers who graduate are equipped to the goals of the curriculum\cite{dk_programming_article}. 

It is a different story regarding the U.K. They implemented programming and computational thinking in the curriculum in 2013, which took effect September 2014\cite{bbc_technology}. To prepare the teachers for this they allocated 1.1 million British Pounds in funding specifically to train school teachers who are new to teaching computing\cite{uk_computing}. This was announced in December 2013 and in February 2014 another $500,000$ British Pounds was allocated in funding to attract businesses to help train teachers\cite{theguardian}.

In the USA they also recognized the need for programming in the schools in 2013\cite{Obama}, but where not willing to spend the money on educating the teachers like in the U.K.
Unlike Denmark however they simply chose not to make programming mandatory yet, but instead focus on preparing the educational system in smaller steps.
They have first focused on making the optional courses in high school programming more attractive by making them count toward the graduation requirements\cite{Code_credit}.
Also recently there have been research done in facilitating the principles of computational thinking in non-programming courses\cite{comp_thinking}.

\section{Discussion}
The United Kingdoms seem to have a good grasp on how to educate teachers with no knowledge about programming. Denmark on the other hand seems to struggle which can possibly lead to problems in the future. To start with, if the teachers are not experienced enough or not enthusiastic about the subject, as it, in the case of Denmark, is the physics/chemistry teacher who mainly has to teach programming, can possibly end up demotivating students from programming. Luckily governments suggests languages such as Scratch to use in lessons, which has a good range of online tutorials and self learning material.

Another possible problem is that if the students are taught in a way that gives them a wrong understanding of aspects in programming or makes them develop bad habits, they can struggle with those problems later in their education, given that they choose an education that is programming related\footnote{These problems are speculative and anecdotal and have no roots in the literature.}. Given these possible problems, it might be interesting to make research in this area to determine if these end up being actual problems.

% 1: http://www.emu.dk/modul/vejledning-faget-matematik
% 2: http://www.version2.dk/artikel/programmering-traeder-ind-i-fysiklokalet-68296
% 3: http://www.bbc.com/news/technology-29010511
% 4: http://www.computing.co.uk/ctg/news/2317331/bcs-given-gbp11m-to-help-teachers-prepare-for-new-computing-curriculum
% 5: http://www.theguardian.com/technology/2014/sep/04/coding-school-computing-children-programming