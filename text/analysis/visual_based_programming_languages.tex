\section{Visual-based Programming Languages}
\label{sec:visual_based_programming_languages}

Traditionally, most programming languages are categorized as text-based because of the way the program logic is written, by making use of a syntax, specific to every language. Therefore, it is often difficult to learn and use a programming language since it requires one to familiarize oneself with the syntax and available constructs first in order to use the language effectively and that takes skill many people lack. 

\subsection{Visual Programming}
In order to address these difficulties, for the past 25 years, research has been done on the so called “Visual Programming” or “Graphical Programming”, and dozens of visual-based programming languages have been created. This approach, reserved and used in the past primarily for systems design, allows the use of spatial representations in two or more dimensions in the form of blocks and different structures and shapes. Compared to text-based programming where lines of code are used, graphical programming replaces these with visual objects, essentially replacing the textual representation of language components with a graphical one, more suitable for visual learners and intuitive for people with no prior knowledge in programming. The creation of programs in such languages is defined by placement and connection between visual objects where the syntax is encoded within the objects' shapes.  
The main aim of visual programming languages (VPL) and environments, as stated by Koitz ans Slany \cite{KoitzSlany14}, is “\textit{diminishing the syntactical burden and enabling a focus on the semantic aspects of coding}.” VPL try to facilitate end-user programming, both kids and adult novice programming, empowering the creation of new programs, not just their consumption, effectively minimising the distance between the cognitive and computational model.

Currently, there is a wide variety of visual programming languages with varying popularity such as Alice, Greenfoot, Tynker, Scratch, Raptor and many more.

\subsection{Scratch}
Scratch is a visual-based programming environment which allows users to create visually-rich, interactive projects.  Since its inception in 2003, the main goal of its creators has been to address the needs and interests of young people (primarily ages 8 to 16) and make for them a soft introduction to the world of programming. Publicly released in 2007, the project has grown in size and scope, with a dedicated site hosting all its 11 million projects and with a user base of 8 million \cite{scratchstat}.
Given its targeted audience, one of the main design goals of Scratch is the focus on self-directed learning and exploration through tinkering with the different constructs of the language and environment. Given this fact and the steady increase of its popularity have prompted hundreds of schools and educational organizations to adopt and integrate it into their curriculum \cite{MaloneyResnick10}.\\
What makes Scratch a sensible choice for people with no prior programming experience is that it has less emphasis on direct instruction than other programming languages and rather focuses on the aspect of learning through self exploration and peer sharing, which breaks the norm of a traditional educational approach.



