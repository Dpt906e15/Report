\chapter{Related Work}
\label{chap:related_work}
The area of educational programming encompasses a somewhat big area of research, with subareas focusing on children (ages 8-16), CS1 courses at university, teaching programming and the transition from novice programming to ``real'' programming. This chapter explores some of the research in these areas.

S. Papert wrote a book in 1980 called ``Mindstorms: Children, Computers, And Powerful Ideas''\cite{s_papert}. He stated in a note that he wanted to demonstrate the fact that computational thinking could be taught to the pupils in the earlier grades of school\cite{turtle_origin}. This shows that the thought of bringing programming, or specifically computational thinking, to children in middle school or earlier, has been around for some time. One of the more recent papers on this subject, by Meerbaum-Salant et al. 2013\cite{learning_computer_scratch}, proposes and evaluates learning material for using Scratch specifically aimed at teachers of middle-school students. Scratch seems like a good choice for this as it is one of the proposed languages to use for the curriculum at least in regards to the UK and Denmark\cite{uk_scratch}\cite{dk_scratch}. The paper concludes that Scratch is a viable platform for teaching CS, but that it should be in combination with close and effective mentoring. Another way of learning e.g. Scratch is through online resources, but according to Lee et al. 2013\cite{ingame_assessment}, many of these resources struggle to keep the users engaged. They propose a way of improving this through assessments integrated into the learning environment. Research is done in both directions of self-directed learning and learning in traditional classes, but focusing on self-directed learning might become less important over time if more and more countries start to add computational thinking to their curriculum.

Even though Scratch is recommended for teaching in middle school, it has elements which makes it cumbersome to work with if the idea is to keep it as part of e.g. math\cite{dk_scratch} as it is not really well suited for more complex formulas. This is shown by Koitz \& Slany, 2014\cite{KoitzSlany14}, who have made a VPL using blocks for coding similar to Scratch but for phones, called PocketCode. One of the major differences they focus on is the fact that PocketCode uses text input for formula manipulation instead of constructing them entirely from blocks. Their reasoning for this is that more complex formulas can get quite big and unmanageable in Scratch, and that it has a problem with how it handles nested blocks. The problem lies in the fact that if a block containing nested blocks is deleted, then the whole thing is deleted. This makes it harder to edit a formula if a user makes a mistake compared to a text-based approach.

Meerbaum-Salant et al. 2015\cite{from_scratch_to_real}, a follow up to their earlier mentioned work, have also researched the transition from novice programming to ``real'' programming. Their research is based around five classes who had a course, where some participants had no programming experience and some had spent a year with Scratch in an earlier grade. They found the participants who had learned Scratch were more motivated and could relate most of what they had experienced in Scratch to the text-based language used. Although they had some wrong impressions about how some constructs worked, they were quickly overcome. They concluded that there were not a significant difference in the grades between the participants who knew Scratch and those who did not, but that knowing Scratch improves the learning of difficult concepts.

In the area of creating text-based programming languages, Stefik \& Siebert, 2013\cite{stefik_all_studies} have developed a language called Quorum. They call it an evidence-based programming language, and the goal is to continually make the syntax and keywords fit what the evidence of current literature suggests. They do this because, claims regarding design decisions in programming languages have often not been backed by scientific evidence in the past\cite{ShaneMarkstrum10}. They have conducted a couple of studies to get an insight into which words and symbols novices find intuitive/not intuitive. Additionally they have conducted empirical studies comparing Quorum to other programming languages with focus on educational purposes. In the comparison they have also used a principle from medicinal science, where they have tried to use a ``placebo'' language together with the other languages chosen. They call it Randomo and it is a somewhat randomly generated language where keywords and symbols are chosen at random from the ASCII table.

The widespread use of electronic devices and media makes the demand for people with digital literacy and skills grow every day. However, there is a trend of school computing lessons on the decline and not enough appropriately qualified teachers to accommodate that demand, as stated by Bundy \& Scott, 2015 \cite{CompThinking}.
In order to tackle the problem, a worldwide movement has been formed in the UK by the Computing at Schools organization (CAS)\cite{CAS}.  It is considered not sufficient to teach children only to program, but also to think computationally and to use abstraction and modularity for understanding and solving problems. Scotland's new Curriculum of Excellence (CfE) provides more freedom for teachers to decide on teaching materials and methods and CS is a core entitlement for  students during the first three years of secondary school.

In 2010, with the help of government bodies, the Royal Society of Edinburgh
(RSE) and the BCS Academy of Computing, Scotland successfully developed a program that combines programming, computational thinking (CT), and evidence-based pedagogy in order to address the computing problem. Due to its success, it was adopted by many other countries. Phase 1 of the program focused on the first three years of secondary school (11-14 in Scotland) relying on a more modernistic approach given the fact that current generations are online,social and very mobile. In terms of programming environments, Scratch and AppInventor were selected because of the way they shift the focus from figuring out the syntax to computational thinking. These environments support the creation of multimedia applications and smartphone apps. Phase 2 of the program focused more on the transition from block-based languages to more traditional text-based ones. The choice was on LiveCode, allowing cross-platform development.



%1: S. Papert, Mindstorms: Children, Computers, And Powerful Ideas, whole book
%2: [11]
%3: Learning computer science concepts with Scratch
%4: UK learning scratch: http://www.computingatschool.org.uk/data/uploads/CASPrimaryComputing.pdf
%5: http://www.emu.dk/modul/vejledning-faget-matematik (eksiterer allerede)
%6: In-Game Assessments Increase Novice Programmers’ Engagement and Level Completion Speed
%7: Empirical Comparison of Visual to Hybrid Formula Manipulation in Educational Programming Languages for Teenagers
%8: From Scratch to “Real” Programming
%9: An Empirical Investigation into Programming....
%10: Staking Claims (eksisterer nok allerede)

%FIX REFS IN TEACHING!!!!!!!!!!!!!!!!!!!!!!!!!!!!!!!!!!