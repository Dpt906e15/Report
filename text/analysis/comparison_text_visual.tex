\section{Comparison}
\label{sec:comparison_text_visual}
As shown in this chapter both text- and visual-based programming languages have parts of their domain dedicated to learning programming. This is done mostly through tools, be it environments for general purpose programming languages, or visual programming languages where the environment can be seen as a part of the language. Both approaches has pros and cons, and these will be explored in this section.

\begin{description}[style=nextline]
\item[Interface Layout] One of the apparent problems with visual programming environments is the fact that a lot of them has multiple elements, e.g. code, available blocks, etc., where each of them requires a slice of the available screen space. This means that a lot more thought should be put into the interface design, where as environments for text-based programming languages can put less focus on it.
\item[Statement Categories] One of the pros of visual programming environments is the fact that they often require a way of dragging e.g. blocks to create functionality. This means that all of the available blocks, or commands in a text based programming language, is always shown and available to the user (it might be split into categories), which gives the user an overview of the possibilities in the language. Text based programming languages lacks in this aspect, as the user often is presented with an empty editor or some tutorial code, which still does not tell the user much about the possibilities. Small Basic is an example where the possibilities are shown through its auto completion, but the user is just presented with a long list and not all of the commands are descriptive in their naming, possibly making it hard to get an overview for a novice.
\item[Writing Speed] One of the pros of text based programming languages is the fact that it is possible to produce code faster compared to visual programming environments. Visual programming environments are often dependent on moving code blocks using the mouse. If a user wants to make changes to e.g. a function made up of multiple blocks, they have to pull it apart to get to the block they wish to exchange. In text based environments the user has the possibility of using both the keyboard and the mouse when they want to select something, giving the possibility of using the preferred peripheral of the user.
\end{description}

From this it can be concluded that visual based programming is better for novices, since the statement categories help overcome the initial writing block, which novices will run into, while the strength of writing speed is not important for them.
Similarly text-based programming is better for experts as the writing speed can improve their productivity, while they do not have any problems knowing what they can do.